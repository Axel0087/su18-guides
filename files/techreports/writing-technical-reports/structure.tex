\section{Structure}

\label{sec:structure}

Beyond the front matter and appendices, your report should contain the
following sections (in order). These sections, and more, is covered in-depth in
the following subsections.

\begin{itemize}

\item Introduction

\item Background

\item Analysis

\item Design

\item Implementation

\item Evaluation

\item Conclusion

\end{itemize}

\subsection{Front Matter and Paging}

The front page of your report should present the reader with a title, list the
course / activity and institution, list the authors, and state the final date
of your report. You may also add author contact information, document and/or
software version information, etc., depending on the context and requirements.

The pages of your report should be numbered, and the headers and/or footers
should provide some contextual information. This is so that if pages are ripped
out of context, they can be stitched back together, in order.

\subsection{Introductory Matters}

A report must begin with a preface and/or abstract stating the purpose, and
scope of your report. It can be a good idea to write this up-front, and to
revise it before submitting your report. It can help you stay on track as you
write your report, and it will tell the reader if reading it is worth their
time and effort.

The preface and/or abstract should be followed by an introduction motivating
the problem, giving a glimpse at the extent of your solution. An introduction
will let the reader know, if reading on is worth further time and effort.

A short report may well begin just there. There is little need for a ``table of
contents'' for a 1--5-page report. For a longer report, the reader may want to
quickly skip to the parts that they find interesting, and so a table of
contents is in order. In either case, it can be a good idea to give an overview
of your report in conclusion of the introduction.

The introduction may also briefly list acknowledgements, if others, beyond the
list of authors, have helped along the way.

\input{\dirpath background}

\input{\dirpath analysis}

\input{\dirpath design}

\input{\dirpath implementation}

\input{\dirpath users-guide-and-examples}

\input{\dirpath evaluation}

\input{\dirpath conclusion}

\input{\dirpath appendices}
